\documentclass[]{article}
\usepackage{lmodern}
\usepackage{amssymb,amsmath}
\usepackage{ifxetex,ifluatex}
\usepackage{fixltx2e} % provides \textsubscript
\ifnum 0\ifxetex 1\fi\ifluatex 1\fi=0 % if pdftex
  \usepackage[T1]{fontenc}
  \usepackage[utf8]{inputenc}
\else % if luatex or xelatex
  \ifxetex
    \usepackage{mathspec}
  \else
    \usepackage{fontspec}
  \fi
  \defaultfontfeatures{Ligatures=TeX,Scale=MatchLowercase}
\fi
% use upquote if available, for straight quotes in verbatim environments
\IfFileExists{upquote.sty}{\usepackage{upquote}}{}
% use microtype if available
\IfFileExists{microtype.sty}{%
\usepackage{microtype}
\UseMicrotypeSet[protrusion]{basicmath} % disable protrusion for tt fonts
}{}
\usepackage[margin=1in]{geometry}
\usepackage{hyperref}
\hypersetup{unicode=true,
            pdftitle={Example Rmd presentation},
            pdfauthor={David},
            pdfborder={0 0 0},
            breaklinks=true}
\urlstyle{same}  % don't use monospace font for urls
\usepackage{graphicx,grffile}
\makeatletter
\def\maxwidth{\ifdim\Gin@nat@width>\linewidth\linewidth\else\Gin@nat@width\fi}
\def\maxheight{\ifdim\Gin@nat@height>\textheight\textheight\else\Gin@nat@height\fi}
\makeatother
% Scale images if necessary, so that they will not overflow the page
% margins by default, and it is still possible to overwrite the defaults
% using explicit options in \includegraphics[width, height, ...]{}
\setkeys{Gin}{width=\maxwidth,height=\maxheight,keepaspectratio}
\IfFileExists{parskip.sty}{%
\usepackage{parskip}
}{% else
\setlength{\parindent}{0pt}
\setlength{\parskip}{6pt plus 2pt minus 1pt}
}
\setlength{\emergencystretch}{3em}  % prevent overfull lines
\providecommand{\tightlist}{%
  \setlength{\itemsep}{0pt}\setlength{\parskip}{0pt}}
\setcounter{secnumdepth}{0}
% Redefines (sub)paragraphs to behave more like sections
\ifx\paragraph\undefined\else
\let\oldparagraph\paragraph
\renewcommand{\paragraph}[1]{\oldparagraph{#1}\mbox{}}
\fi
\ifx\subparagraph\undefined\else
\let\oldsubparagraph\subparagraph
\renewcommand{\subparagraph}[1]{\oldsubparagraph{#1}\mbox{}}
\fi

%%% Use protect on footnotes to avoid problems with footnotes in titles
\let\rmarkdownfootnote\footnote%
\def\footnote{\protect\rmarkdownfootnote}

%%% Change title format to be more compact
\usepackage{titling}

% Create subtitle command for use in maketitle
\newcommand{\subtitle}[1]{
  \posttitle{
    \begin{center}\large#1\end{center}
    }
}

\setlength{\droptitle}{-2em}

  \title{Example Rmd presentation}
    \pretitle{\vspace{\droptitle}\centering\huge}
  \posttitle{\par}
    \author{David}
    \preauthor{\centering\large\emph}
  \postauthor{\par}
      \predate{\centering\large\emph}
  \postdate{\par}
    \date{2019-10-20}


\begin{document}
\maketitle

\hypertarget{section-header-here}{%
\section{Section Header here}\label{section-header-here}}

\hypertarget{here-is-a-slide}{%
\subsection{Here is a slide}\label{here-is-a-slide}}

here is some text

\hypertarget{here-is-another-slide}{%
\subsection{Here is another slide}\label{here-is-another-slide}}

\hypertarget{here-are-some-bullets}{%
\subsection{Here are some bullets}\label{here-are-some-bullets}}

\begin{itemize}
\item
  bullet 1
\item
  bullet 2
\end{itemize}

\hypertarget{lets-do-bullets-incrementally}{%
\subsection{Lets do bullets
incrementally}\label{lets-do-bullets-incrementally}}

\begin{quote}
\begin{itemize}
\tightlist
\item
  bullet 1
\end{itemize}
\end{quote}

\begin{quote}
\begin{itemize}
\tightlist
\item
  bullet 2
\end{itemize}
\end{quote}

\hypertarget{another-way}{%
\subsection{Another way}\label{another-way}}

\begin{itemize}
\item
  bullet
\item
  bullet
\end{itemize}

\hypertarget{loading-in-data}{%
\section{Loading in data}\label{loading-in-data}}

\hypertarget{just-as-normal}{%
\subsection{Just as normal}\label{just-as-normal}}

-RELATIVE PATHS

\begin{quote}
\begin{itemize}
\tightlist
\item
  YOU ARE NOT WORKING ALONE
\end{itemize}
\end{quote}

\hypertarget{working-on-data}{%
\subsection{Working on data}\label{working-on-data}}

\begin{itemize}
\tightlist
\item
  Do it normally
\end{itemize}

\hypertarget{plots}{%
\section{Plots}\label{plots}}

\hypertarget{plots-1}{%
\subsection{Plots}\label{plots-1}}


\end{document}
